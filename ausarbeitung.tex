% ----------------------------------------------------------------------------
%
% ----------------------------------------------------------------------------

\documentclass[11pt, parskip=half]{scrartcl}       % KOMA-Skript für Artikel

%% Präambel
\usepackage[english, ngerman]{babel} % deutsche typogr. Regeln + Trenntabelle
\usepackage[T1]{fontenc}             % interner TeX-Font-Codierung
\usepackage{lmodern}                 % Font Latin Modern
\usepackage[utf8]{inputenc}          % Font-Codierung der Eingabedatei
\usepackage[babel]{csquotes}         % Anführungszeichen
\usepackage{graphicx}                % Graphiken
\usepackage{booktabs}                % Tabellen schöner
\usepackage{amsmath}      % Mathematik
\usepackage{amssymb}      % Mathematische Symbole
\usepackage{float}
\usepackage[pdftex]{hyperref}
\usepackage{subcaption}
\usepackage{url}
\hypersetup{
  bookmarksopen=true,
  bookmarksopenlevel=3,
  colorlinks,
  citecolor=blue,
  linkcolor=blue
}
\usepackage{scrhack} % unterdrückt Fehlermeldung von listings
\usepackage[backend=bibtex]{biblatex}
\addbibresource{/home/niklas/Documents/bibfile/bibliographie.bib}


%% Nummerierungstiefen
\setcounter{tocdepth}{3}     % 3 Stufen im Inhaltsverzeichnis
\setcounter{secnumdepth}{3}  % 3 Stufen in Abschnittnummerierung

\usepackage[nolist]{acronym}
\begin{acronym}
\acro{pwa}[PWA]{Progressive Web App}
\end{acronym}

\newcommand\litem[1]{\item{\bfseries#1.\space}}

\newcommand*{\N}{\mathbb{N}}
\newcommand*{\Z}{\mathbb{Z}}
\newcommand*{\Q}{\mathbb{Q}}
\newcommand*{\R}{\mathbb{R}}

% Code packages
\usepackage{listingsutf8} % Code mit UTF-8 Support
\usepackage{color,xcolor} % Farben definierbar als HTML, RGB, ...
\usepackage{textcomp}

\definecolor{alabasterGray}{HTML}{F7F7F7}
\definecolor{alabasterBlue}{HTML}{325CC0}
\definecolor{alabasterGreen}{HTML}{448C27}
\definecolor{alabasterPink}{HTML}{7A3E9D}
\definecolor{alabasterRed}{HTML}{AA3731}

\lstset{
  basewidth={0.5em,0.45em},
  extendedchars=true,
  backgroundcolor=\color{alabasterGray}, % background color (color or xcolor needed)
  basicstyle=\small\ttfamily,
  keywordstyle=\color{alabasterBlue},
  commentstyle=\color{alabasterRed},
  rulecolor=\color{black},          % frame color may change if not set (frame on comment, is comment-colored)
  stringstyle=\color{alabasterGreen},
  numberstyle=\tiny\color{gray},    %
  numbers=none,                     % where to put the line-numbers (none, left, right)
  numbersep=7pt,                    % margin between numbers and code
  stepnumber=1,                     % every n-th row will be numbered
  captionpos=b,                     % sets the caption-position to bottom
  frame=single,                     % adds a frame around the code
  keepspaces=true,                  % keeps spaces in text
  showtabs=false,                   % show tabs as special char
  showspaces=false,                 % show spaces as special char
  showstringspaces=false,           % show spaces in strings only as special char
  tabsize=2,                        % tabsize is 2 spaces
  breakatwhitespace=false,          % sets if automatic breaks should only happen at whitespace
  breaklines=true,                  % sets automatic line breaking
}
\lstset{
  literate={ö}{{\"o}}1
           {ä}{{\"a}}1
           {ü}{{\"u}}1
           % https://tex.stackexchange.com/questions/17739/listings-package-how-to-highlight-math-operators
           {true}{{{\color{alabasterPink}true}}}4
           {false}{{{\color{alabasterPink}false}}}5
           {TRUE}{{{\color{alabasterPink}TRUE}}}4
           {FALSE}{{{\color{alabasterPink}FALSE}}}5
           {True}{{{\color{alabasterPink}True}}}4
           {False}{{{\color{alabasterPink}False}}}5
}
\expandafter\def\expandafter\UrlBreaks\expandafter{\UrlBreaks\do\-\do\\/}

\begin{document}

\titlehead{\includegraphics[width=\textwidth]{src/Logo_THM_CG_FB06.png}}

%% Titelseite
\subject{Manusskript}
\title{Progressive Web Apps}
\subtitle{Kurs \enquote{Hauptseminar -- Mobile Techonologies}}
\author{Niklas Deworetzki}
\date{\today}
\maketitle

\vspace*{1.5cm}
\section*{\centerline{Zusammenfassung}}

Diese Arbeit befasst sich mit Progressive Web Applications.


\newpage

\tableofcontents
\newpage

\section{Einleitung}

https://medium.com/@amberleyjohanna/seriously-though-what-is-a-progressive-web-app-56130600a093

\acp{pwa} 
Progressive Web Apps werden immer relevanter.
Ergänzendes Element zwischen herkömmlicher Anwendung und Website.
Was kann man damit machen? Worum geht's eigentlich?

\section{Was ist eine PWA?}

Im Folgenden wird zunächst versucht, zu klären, was eine \ac{pwa} ist.
Da es keine offizielle Definition für \acp{pwa} gibt, werden an dieser Stelle verschiedene Meinungen, Definitionen und Empfehlungen zusammengetragen.
Diese lassen sich dabei in drei Kategorien einteilen:

\begin{enumerate}
\litem{Grundsätzliche Eigenschaften} Eigenschaften einer \ac{pwa}, welche grundlegende Funktionen ermöglichen.
\litem{Verbreitete Eigenschaften} Eigenschaften einer \ac{pwa}, welche von verschiedenen Quellen als Voraussetzung genannt werden, aber nicht unbedingt notwendig für das Funktionieren der \ac{pwa} sind.
\litem{Designempfehlungen} Eigenschaften einer \ac{pwa}, die sich positiv auf das Nutzererlebnis auswirken.
\end{enumerate}


\subsection{Grundsätzliche Eigenschaften}

Eine der wenigen Eigenschaften einer \ac{pwa}, über die Einigkeit herrscht, ist, dass eine \ac{pwa} installierbar sein muss.
Dies bedeutet, dass es dem Nutzer möglich sein soll, die \ac{pwa} nach einem Installationsvorgang direkt vom Gerät aus starten zu können.
Ob dies nun über einen Desktopeintrag oder etwa eine Verknüpfung auf dem Startbildschirm erfolgt, ist abhängig vom Gerät des Benutzers.
Wichtig ist bloß, dass sich die installierte \ac{pwa} wie eine herkömmlich-installierte native Anwendung verhalten soll.

Dabei ist es natürlich nötig, Unterstützung vom Browser zu erhalten, welcher die \ac{pwa} startet, ausführt und anzeigt.
Dieser übernimmt auch die Verwaltung der Installation und fügt die entsprechenden Einträge in Menüs oder dem Desktop hinzu.
Um dem Nutzer ein mit einer herkömmlichen App vergleichbares Erlebnis zu bieten, erhalten \acp{pwa} beim Ausführen auch eine Sonderbehandlung vom Browser.
In Google-Chrome beispielsweise wird jede \ac{pwa} in einem eigenen Fenster gestartet, wodurch das Nutzererlebnis einer nativen Anwendung entstehen soll\cite{googledevs_pwa}.


Durch den Installationsprozess werden Teile der Anwendung lokal auf dem Gerät gespeichert.
Die Auswahl und Verwaltung der lokal gespeicherten Inhalte der Anwendung geschieht über sogenannte \enquote{Service Worker}, welche bei der Installation konfiguriert werden.
Darüber, dass das Speichern von Anwendungsinhalten durch Service Worker für eine \ac{pwa} notwendig ist, sind sich die meisten Quellen einig.
Jedoch werden verschiedene Gründe für das Speichern angeführt.
Laut dem Google-Entwickler Pete LePage optimiert das lokale Speichern von Anwendungsteilen den Startvorgang einer \ac{pwa}.
Durch das Eliminieren von Engpässen bei der Netzwerkgeschwindigkeit könne eine konsistent schnelle Ladezeit sichergestellt werden.\cite{googledev_pwaondesktop}
Bei Mozilla ist das lokale Speichern von Anwendungsinhalten unter dem Punkt der Netzwerkunabhängigkeit betrachtet.
Laut Chris David Mills solle so die Anwendung nicht nur bei schlechter Netzwerkverbindung funktionstüchtig bleiben, sondern gar ohne eine Netzwerkverbindung auskommen können\cite{mozilladevs-pwaintroduction}.

Durch die Verwendung von Service Worker ergibt sich eine weitere Eigenschaft aller \acp{pwa}.
Um Service Worker verwenden zu können, muss eine Website nämlich über % TODO


\subsection{Häufige Eigenschaften}

\subsection{Designmerkmale}


\section{Technische Voraussetzungen}

\subsection{Installationsformat}

\subsection{Datenverwaltung durch Service Worker}

\subsection{Interaktion mit Nutzern}

\subsection{Interaktion mit Hardware}

\subsection{Browserspezifische Voraussetzungen}


\section{Vorteile von PWAs}

\section{Kritik an PWAs}


\section{Fazit}

\newpage

\printbibliography

\end{document}

%%% Local Variables:
%%% mode: latex
%%% TeX-master: t
%%% End:
